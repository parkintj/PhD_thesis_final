\pagestyle{fancy}
\headheight 20pt
\lhead{Ph.D. Thesis --- T.~J. Parkin}
\rhead{McMaster - Physics \& Astronomy}
\chead{}
\lfoot{}
\cfoot{\thepage}
\rfoot{}
\renewcommand{\headrulewidth}{0.1pt}
\renewcommand{\footrulewidth}{0.1pt}

\chapter{Summary \& Future Work} \label{chapter5}
\thispagestyle{fancy}

In this thesis I have utilised the unprecedented sensitivity and resolution of the \emph{Herschel Space Observatory}, designed to probe the cold interstellar medium (ISM) from roughly 55 to 672~$\mu$m using both photometric and spectroscopic instruments.  Our goal was to search for local variations within the ISM of two contrasting environments, namely the nearby galaxies Centaurus~A (Cen~A) and M51.  In addition, I strived to contribute to the overall understanding of the effects of an active galactic nucleus (AGN) to the neighbouring ISM.  Lastly, I wanted to characterise the nature of the disk of Cen~A to establish if it warrants the `peculiar' designation it has received in its morphological classification.

In Chapter~\ref{chapter2} I focused on the cold gas and dust in Cen~A using \emph{Herschel} PACS and SPIRE observations of the dust continuum at 70, 160, 250, 350 and 500~$\mu$m.  I combined these data with observations of the CO($J=3-2$) transition from the HARP-B instrument on the James Clerk Maxwell Telescope to obtain measurements of the molecular gas content of the warped disk.  Using a modified blackbody I modelled the dust spectral energy distribution (SED) on a pixel-by-pixel basis with the \emph{Herschel} photometry and found a radial decrease in both the dust temperature, varying from 30~K down to 20~K, as well as the dust mass distribution.  Furthermore, I found a similar radial distribution in the total gas distribution.  Most interestingly, however, when I created a map of the gas-to-dust mass ratio I found a radial trend in this as well.  While the average value I found of $103 \pm 8$ is consistent with that of the Milky Way, the region in the vicinity of the AGN had a gas-to-dust ratio of almost 275.  Eliminating a gradient in the metallicity or the CO($J=3-2$)/CO($J=1-0$) ratio as possible explanations for this trend, I concluded that the AGN is removing dust from the surrounding ISM via dust sputtering by a hard X-ray radiation field or dust expulsion by the jets.  Thus, in the case of Cen~A I have found that indeed the AGN does have an effect on the neighbouring regions.

Next, I turned to the grand-design spiral galaxy M51 and presented \emph{Herschel} spectroscopy of the central $2.5\arcmin \times 2.5\arcmin$ of the disk in Chapter~\ref{chapter3}.  These observations targeted the important far-infrared fine-structure lines [C~\textsc{ii}](158~$\mu$m), [N~\textsc{ii}](122 \& 205~$\mu$m), [O~\textsc{i}](63 \& 145~$\mu$m) and [O~\textsc{iii}](88~$\mu$m), which are the dominant contributors to the global gas cooling budget in both neutral and ionised phases, particularly in photon dominated (photodissociation) regions (PDRs) and H~\textsc{ii} regions.  I subdivided our maps into four distinct regions, namely the nucleus (which contains a Seyfert~2 nucleus), centre, arm and interarm regions, to search for trends in the physical characteristics of the gas.  I determined for the first time that there is a radial trend in the contribution to the [C~\textsc{ii}](158~$\mu$m) emission from ionised gas, from 80\% in the nucleus down to about 50\% in the arm and interarm regions.  Furthermore, I found a slight suppression in the heating efficiency in the nucleus compared to the other regions, as shown by a decrease in the ([C~\textsc{ii}](158~$\mu$m)+[O~\textsc{i}](63~$\mu$m))/$F_{\mathrm{TIR}}$ ratio with increasing 70~$\mu$m/160~$\mu$m colour.  A comparison between our spectroscopy and a PDR model revealed a decreasing trend in the strength of the far-ultraviolet radiation field, $G_{0}$ and density of hydrogen nuclei, $n$, as a function of increasing radius (or region).  For the first time I also show that there is no difference in $G_{0}$ and $n$ between the arm and interarm regions, despite having different star formation rate densities.  This in turn, indicates that there is no difference in the molecular cloud properties between those in the spiral arms of M51, and those in the interarm regions.  Again I have shown that an active nucleus affects the ISM in the surrounding region.

Lastly, in Chapter~\ref{chapter4} I examine a radial strip of the eastern half of the disk in Cen~A in the same cooling lines I investigated in M51.  I divided the strip into eight bins to deduce any radial trends in the gas component of the ISM, as I found in M51 in Chapter~\ref{chapter2}.  I found that there is a slight trend with increasing radius in the heating efficiency, as it is reduced near the nucleus by about a factor of two compared to larger radii.  I also determined that, unlike M51, the majority of the [C~\textsc{ii}](158~$\mu$m) emission originates in neutral gas.  Futhermore, I also do not observe any significant radial changes in the values of $G_{0}$ and $n$.  However, the most significant result of this investigation is that the values of $G_{0}$ and $n$ are more consistent with normal spiral galaxies than elliptical galaxies.  Thus, while Cen~A is a typical elliptical galaxy on global scales, its disk exhibits properties of a spiral galaxy with an active nucleus.

There are a few implications of the results of this thesis on the evolution of the ISM in general.  A reduction in the heating efficiency has been attributed to the dust grains and polycyclic aromatic hydrocarbons becoming too positively charged to efficiently free electrons, which in turn, contribute to gas heating \citep{1985ApJ...291..722T,2012ApJ...747...81C,2012A&A...548A..91L}.  I observe such a decrease in the nuclei of both M51 and Cen~A, though in Cen~A it is slightly less obvious.  This would imply that the active nuclei of both sources are impacting the  heating and cooling processes in the gas, which ultimately will lead to a reduction in the gas's ability to cool enough to form molecular cores and the next generation of stars.

My results in Chapter~\ref{chapter3} and Chapter~\ref{chapter4} demonstrate that molecular clouds possess similar properties regardless of whether they are in the arms or interarm regions.  Ultimately, this result would imply that the reason the spiral arms appear more prominent in disk galaxies is simply due to an increase in the number of molecular clouds producing young stars that, in turn, contribute to heating the gas and dust.  Perhaps molecular clouds are also very similar in nature in all galaxies producing stars, and this is something I can look into by studying other nearby galaxies.

\subsection{Future Work}\label{future}
To further solidify our conclusions that active galactic nuclei directly affect the surrounding ISM, I would like to extend our investigation to other nearby galaxies.  In particular, it would be ideal to probe sources with varying AGN luminosities to see if their impact on the ISM changes with luminosity.  A couple of potential options are NGC~4151 and NGC~1068.  Both of these galaxies have been observed with \emph{Herschel} as part of the Very Nearby Galaxies Survey.  In addition, I have obtained CO($J=3-2$) observations from the JCMT, revealing detections in CO($J=3-2$) for NGC~1068, but interestingly, I only determined an upper limit for NGC~4151.  A previous study of NGC~1068 by \citet{2012ApJ...758..108S} used the SPIRE FTS instrument to study the CO ladder and determined there is an X-ray dominated region likely influenced by the AGN.  We can expand on this work to fully probe the surrounding ISM.  NGC~4151 is an interesting target because we do not observe CO($J=3-2$), thus suggesting there is very little warm molecular gas in it; however, it has been detected in CO($J=1-0$) \citep{2010ApJ...721..911D}.  It has a Seyfert~1 nucleus \citep{2000A&ARv..10..135U} and is classified as a barred spiral \citep{1991trcb.book.....D}.  Thus, this galaxy would give us yet another unique perspective on AGN impact on the ISM.

Another potential direction we can take is to expand our PDR region modelling to other galaxies to further characterise the molecular cloud properties.  One way to definitively explain the subtle differences we see between Cen~A and M51 is to investigate a target that is fully edge on, such as NGC~891.  This galaxy is classified as an SAb morphological type \citep{1991trcb.book.....D}, and is located about 9.6~Mpc away \citep{2004ApJS..151..193S}, giving us the same spatial resolution as M51.  It is believed to be a close analogue to the Milky Way, and its orientation allows for in-depth studies of the disk and the halo \citep{1984A&A...140..470V,1993ApJ...404L..59S,2009MNRAS.395...97W}.  It has also been well studied at numerous wavelengths \citep[e.g.][]{1993ApJ...404L..59S,2009MNRAS.395...97W,2011A&A...531L..11B,2013ApJ...762...12H}.  Recently, \citet{2013ApJ...766...57B} studied the gas in the halo 5~kpc above the plane and determined from its metallicity that it was  from the disk via a galactic fountain.  These properties make it an interesting target for conducting further PDR analysis.

We have spectroscopic observations of a radial strip along one half of the disk of NGC~891 in the same important cooling lines as we have just presented for M51 and Cen~A from PACS and SPIRE.  A similar analysis into NGC~891 could aid in interpreting our results in Chapters~\ref{chapter3} and \ref{chapter4}, and determine if the slight difference in $G_{0}$ and $n$ between the two galaxies is an inclination effect.  

Lastly, I can propose to observe Cen~A with the Atacama Large Millimetre Array (ALMA) to aid in contraining our PDR model results.  ALMA will be able to observe Cen~A at unprecedented resolution in the submillimetre, giving us access to the CO($J=3-2$) line and its adjacent dust continuum on spatial scales of less than 1.0--1.4$\arcsec$ \citep{alma_primer}, corresponding to physical scales of less than approximately 26~pc at the distance of Cen~A.  At these scales, we can resolve individual giant molecular clouds with masses larger than roughly $10^{4}$~M$_{\odot}$, and possibly resolve the nucleus, which is still point-source like in a 2.4$\arcsec \times 6\arcsec$ beam \citep{2009ApJ...695..116E}.

It is crucial to take advantage of infrared and submillimetre observations of nearby galaxies to help us fully understand the life cycle of the ISM and the impact of star formation on its host molecular cloud.  Now that the \emph{Herschel Space Observatory} has ended its observing lifetime, we need to fully exploit the data it has produced and continue to probe the local properties of the dust and gas in nearby galaxies.  If we can understand the evolution of the ISM we can improve our understanding on galaxy evolution as a whole.  This thesis has presented important results that aid in bringing us closer to this goal.

%\bibliography{thesis_bib}
%\bibliographystyle{apj}

\begin{thebibliography}{17}
\expandafter\ifx\csname natexlab\endcsname\relax\def\natexlab#1{#1}\fi

\bibitem[{{Bianchi} \& {Xilouris}(2011)}]{2011A&A...531L..11B}
{Bianchi}, S., \& {Xilouris}, E.~M. 2011, \aap, 531, L11

\bibitem[{{Bregman} {et~al.}(2013){Bregman}, {Miller}, {Seitzer}, {Cowley}, \&
  {Miller}}]{2013ApJ...766...57B}
{Bregman}, J.~N., {Miller}, E.~D., {Seitzer}, P., {Cowley}, C.~R., \& {Miller},
  M.~J. 2013, \apj, 766, 57

\bibitem[{{Croxall} {et~al.}(2012){Croxall}, {Smith}, {Wolfire}, {Roussel},
  {Sandstrom}, {Draine}, {Aniano}, {Dale}, {Armus}, {Beir{\~a}o}, {Helou},
  {Bolatto}, {Appleton}, {Brandl}, {Calzetti}, {Crocker}, {Galametz}, {Groves},
  {Hao}, {Hunt}, {Johnson}, {Kennicutt}, {Koda}, {Krause}, {Li}, {Meidt},
  {Murphy}, {Rahman}, {Rix}, {Sauvage}, {Schinnerer}, {Walter}, \&
  {Wilson}}]{2012ApJ...747...81C}
{Croxall}, K.~V., {Smith}, J.~D., {Wolfire}, M.~G., {et~al.} 2012, \apj, 747,
  81

\bibitem[{{de Vaucouleurs} {et~al.}(1991){de Vaucouleurs}, {de Vaucouleurs},
  {Corwin}, {Buta}, {Paturel}, \& {Fouque}}]{1991trcb.book.....D}
{de Vaucouleurs}, G., {de Vaucouleurs}, A., {Corwin}, Jr., H.~G., {et~al.}
  1991, {Third Reference Catalogue of Bright Galaxies} (Springer-Verlag, New
  York)

\bibitem[{{Dumas} {et~al.}(2010){Dumas}, {Schinnerer}, \&
  {Mundell}}]{2010ApJ...721..911D}
{Dumas}, G., {Schinnerer}, E., \& {Mundell}, C.~G. 2010, \apj, 721, 911

\bibitem[{{Espada} {et~al.}(2009){Espada}, {Matsushita}, {Peck}, {Henkel},
  {Iono}, {Israel}, {Muller}, {Petitpas}, {Pihlstr{\"o}m}, {Taylor}, \&
  {Dinh-V-Trung}}]{2009ApJ...695..116E}
{Espada}, D., {Matsushita}, S., {Peck}, A., {et~al.} 2009, \apj, 695, 116

\bibitem[{{Hodges-Kluck} \& {Bregman}(2013)}]{2013ApJ...762...12H}
{Hodges-Kluck}, E.~J., \& {Bregman}, J.~N. 2013, \apj, 762, 12

\bibitem[{{Lebouteiller} {et~al.}(2012){Lebouteiller}, {Cormier}, {Madden},
  {Galliano}, {Indebetouw}, {Abel}, {Sauvage}, {Hony}, {Contursi}, {Poglitsch},
  {R{\'e}my}, {Sturm}, \& {Wu}}]{2012A&A...548A..91L}
{Lebouteiller}, V., {Cormier}, D., {Madden}, S.~C., {et~al.} 2012, \aap, 548,
  A91
  
\bibitem[\protect\citeauthoryear{ALMA Primer}{2012}]{alma_primer}
  {Schieven}, G., ed., 2012, Observing with ALMA: A Primer for Early Science, ALMA Doc. 		1.1, ver. 1.1

\bibitem[{{Scoville} {et~al.}(1993){Scoville}, {Thakkar}, {Carlstrom}, \&
  {Sargent}}]{1993ApJ...404L..59S}
{Scoville}, N.~Z., {Thakkar}, D., {Carlstrom}, J.~E., \& {Sargent}, A.~I. 1993,
  \apjl, 404, L59

\bibitem[{{Spinoglio} {et~al.}(2012){Spinoglio}, {Pereira-Santaella},
  {Busquet}, {Schirm}, {Wilson}, {Glenn}, {Kamenetzky}, {Rangwala}, {Maloney},
  {Parkin}, {Bendo}, {Madden}, {Wolfire}, {Boselli}, {Cooray}, \&
  {Page}}]{2012ApJ...758..108S}
{Spinoglio}, L., {Pereira-Santaella}, M., {Busquet}, G., {et~al.} 2012, \apj,
  758, 108

\bibitem[{{Strickland} {et~al.}(2004){Strickland}, {Heckman}, {Colbert},
  {Hoopes}, \& {Weaver}}]{2004ApJS..151..193S}
{Strickland}, D.~K., {Heckman}, T.~M., {Colbert}, E.~J.~M., {Hoopes}, C.~G., \&
  {Weaver}, K.~A. 2004, \apjs, 151, 193

\bibitem[{{Struve} {et~al.}(2010){Struve}, {Oosterloo}, {Morganti}, \&
  {Saripalli}}]{2010A&A...515A..67S}
{Struve}, C., {Oosterloo}, T.~A., {Morganti}, R., \& {Saripalli}, L. 2010,
  \aap, 515, A67+

\bibitem[{{Tielens} \& {Hollenbach}(1985)}]{1985ApJ...291..722T}
{Tielens}, A.~G.~G.~M., \& {Hollenbach}, D. 1985, \apj, 291, 722

\bibitem[{{Ulrich}(2000)}]{2000A&ARv..10..135U}
{Ulrich}, M.-H. 2000, \aapr, 10, 135

\bibitem[{{van der Kruit}(1984)}]{1984A&A...140..470V}
{van der Kruit}, P.~C. 1984, \aap, 140, 470

\bibitem[{{Whaley} {et~al.}(2009){Whaley}, {Irwin}, {Madden}, {Galliano}, \&
  {Bendo}}]{2009MNRAS.395...97W}
{Whaley}, C.~H., {Irwin}, J.~A., {Madden}, S.~C., {Galliano}, F., \& {Bendo},
  G.~J. 2009, \mnras, 395, 97

\end{thebibliography}



\label{lastpage}
