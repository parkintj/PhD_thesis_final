\thispagestyle{fancy}

\textrm{}\\\\
\noindent\textbf{\huge\textsf{Abstract}}\\\\

\noindent I investigate the interstellar medium (ISM) of two nearby, resolved galaxies, M51 (NGC~5194) and Centaurus~A (Cen~A; NGC~5128) using spectroscopic and photometric data from the \emph{Herschel Space Observatory} to search for local variations of the characteristics of the ISM.  I find that the average characteristics of the ISM in Cen~A, a giant elliptical galaxy, are similar to those typically found in normal star forming galaxies, despite its unique morphology and classification as a radio galaxy with an active galactic nucleus (AGN).  Using \emph{Herschel} photometry I find radial trends in the dust temperature, the dust mass, and unexpectedly, the gas-to-dust mass ratio.  I hypothesise that the AGN is removing nearby dust grains via dust sputtering or expulsion via jets.  A comparison of \emph{Herschel} spectroscopy of important cooling lines to a photon dominated region (PDR) model reveals the strength of the far-ultraviolet radiation field, $G_{0}$, and the hydrogen gas density, $n$, in the PDR gas within Cen~A are consistent with values found in other nearby galaxies.  I do not observe any obvious radial trend in these characteristics and conclude that the high inclination of Cen~A may be inhibiting the identification of any impact the AGN is having on the surrounding gas.

An investigation of the gas in M51 using a similar spectroscopic dataset as with Cen~A shows for the first time that a large fraction of the observed [C~\textsc{ii}](158~$\mu$m) emission in the centre of M51 originates in diffuse ionised gas.  This fraction falls off with radius out to the arm and interarm regions.  I also find via PDR modelling that there is a decreasing radial trend in the values of $G_{0}$ and $n$, and that in the arm and interarm regions they are the same.  Thus, there appears to be no difference in the physical properties of the molecular clouds in the arm and interarm regions of the galaxy.  The results of this thesis contribute to further understanding the characteristics of the ISM of nearby galaxies, as well as the evolution of the ISM in galaxies containing an AGN.


\newpage
\thispagestyle{empty}
\mbox{}
%to get the Dedication of the right-hand side